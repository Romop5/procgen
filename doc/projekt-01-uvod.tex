
%/////////////////////////////////////////////////
%OSNOVA BP PRACE (XDOBIA11)
%/////////////////////////////////////////////////

%01 Uvod
%====
\chapter{Úvod}
%	- k comu je mozne pouzit proc. gen (motivacia)
Táto práca sa zaoberá procedurálnym generovaním a vytvorením nového programovacieho jazyka k nemu určenému.

Procedurálne generovanie obecne predstavuje syntézu dát, ktorá je popísaná algoritmicky (procedúrami).
V praxi je možné tento prístup využiť napríklad k tvorbe komplexných a detailných dát.

%TODO vyuzitie a k comu je to dobre
%%%%%vytvárať vizuálne bohatý obsah na základe jednoduchých popisov. 
%	- kde je mozne taketo nieco vyuzit (hry, dema, kultura)
Podobné techniky v grafike majú svoje opodstatnenie. V počiatkoch prvých video hier našli svoje uplatnenie v generovaní obsahu, textúr a iný herných prvkov z dôvodu pamäťových obmedzení. 
V súčasnej dobe sa ich využitie spája s generovaním rozsiahlych trojrozmerných herných svetov.
%	- preco k tomu porebujeme prave jazyk (komfort, elegancia)

Pre využitie procedurálneho generovania je nutné zostrojiť systém, ktorý by autonómne vykonával generovanie z popisov. K tomuto účelu je možné
použiť obvyklé programovacie jazyky. Programovanie v týchto jazykoch ale ovykle obnáša nutnosť riešiť problémy ako sú správa pamäte a podobne, čo znižuje komfort programátora.
Preto má zmysel uvažovať o vytvorení nového programovacieho jazyka, ktorý umožní tvoriť programy bližšie abstrakcii procedurálneho generovania 
a tým zlepší komfort.

%	- o čom je tato BP ?
%	=> chceme vytvarat rozsiahle podla nejakych pravidiel data na zaklade roznych vztahov
%	=> k tomu potrebujeme: nastroj ktory nam umozni taketo generovanie realizovat:
%		=> co najkomfortnejsie
%		=> pripadne efektivne

%	- preto je nutne zmienit principy, na ktorych je generovanie zalozene aby:
%		=> pripadny uzivatel lepsie pochopil co sa deje vo vnutri
%	- taktiez je vhodne spomenut principy pomocou ktorych sa navrhuju poc. jazyky
%		=> co to je gramatika, prekladac
%	- nakoniec popiseme aky jazyk sme navrhli a co jeho navrhom riesime (resp. blizsie riesime)

V prvej kapitole sú preto predstavené zaužívané formálne štruktúry a prístupy, ktoré sa obecne používajú k proc. gen.

V kapitole XYZ sú všeobecne popísané formálne nástroje na ktorých sú obecne založené programovacie jazyky.

V kapitole N nakoniec opisujeme návrh jazyka.














