
%/////////////////////////////////////////////////
%OSNOVA BP PRACE (XDOBIA11)
%/////////////////////////////////////////////////

%01 Uvod
%====
\chapter{Úvod}
%	- k comu je mozne pouzit proc. gen (motivacia)
Táto práca sa zaoberá procedurálnym generovaním v oblasti počítačovej grafiky.
Procedurálne generovanie predstavuje jeden zo spôsobov akým je možné vytvárať vizuálne bohatý obsah na základe jednoduchých popisov. 
%	- kde je mozne taketo nieco vyuzit (hry, dema, kultura)
Podobné techniky v grafike majú svoje opodstatnenie. V počiatkoch prvých video hier našli svoje uplatnenie v generovaní obsahu, textúr a iný herných prvkov z dôvodu pamäťových obmedzení. 
V súčasnej dobe sa ich využitie spája s generovaním rozsiahlych trojrozmerných herných svetov.
%	- preco k tomu porebujeme prave jazyk (komfort, elegancia)

Pre využitie procedurálneho generovania je nutné zostrojiť systém, ktorý by autonómne vykonával generovanie z popisov. K tomuto účelu je možné
použiť obvyklé programovacie jazyky. Programovanie v týchto jazykoch ale ovykle obnáša nutnosť riešiť problémy ako sú správa pamäte a podobne, čo znižuje komfort programátora.
Preto má zmysel uvažovať o vytvorení nového programovacieho jazyka, ktorý umožní komfortne programovať.


