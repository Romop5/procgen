%02 Teoria - Proc generovanie
%=========================
\chapter{L systémy}
%	L-SYSTEMY
%	=========
L-systémy boli pôvodne predstavené v roku 1968 matematikom Lindenmayerom ako matematický formalizmus pre popis rastu a počítačovú vizualizáciu rastlín \cite{Lsystems}.
Ich definícia vychádza z prepisovacích systémov, ktoré si predstavíme v podkapitole \ref{rewriting}.
Napriek ich jednoduchosti umožňujú vytvárať komplexné štruktúry. V počítačovej grafike sa obvykle využívajú spolu s tzv. korytnačiou grafikou (angl. turtle graphics).

Existuje niekoľko rozšírení L-systémov, ktoré umožňujú elegantnejšie popísať zložité štruktúry. Jedným z nich sú parametrické L-systémy, ktoré si predstavíme v podkapitole
\ref{parametric}.

\section{Prepisovacie systémy}\label{rewriting}
\section{DOL L-systémy}\label{dollsystems}
\section{Parametrické L-systémy}\label{parametric}
\section{Šumy}
%	SUMY (NOISE)
%	============
		- popis zakladnych sumov, ktore sa vyuzivaju v proc.gen


