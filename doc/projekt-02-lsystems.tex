\chapter{Procedurálne generovanie}

V tejto kapitole je popísané samotné generovanie, jeho zmysel a využitie, a niektoré obvyklé metódy, pomocou ktorých sa v praxi realizuje.
\section{Definícia}
Formálne je možno definovať procedurálne generovanie ako spôsob vytváranie obsahu na základe algoritmov  bez alebo s obmedzeným ľudským zásahom.
\footnote{\url{https://en.wikipedia.org/wiki/Procedural_generation}} 

V praxi sa jedná o automatizované vytvorenie obsahu, ktorý vzniká z predom definovaného vstupu pomocou postupov (algoritmov), ktoré sú riadene parametrami.

Pod slovom obsah je obecne možné dosadiť akýkoľvek prejav ľudského intelektu, od textu, obrazu až po hudbu, ktorý by pri klasickej tvorbe bol vytvorený manuálne človekom.

Výhody tohoto prístupu teda spočívajú v:
\begin{itemize}
\item možnosti získania rozsiahlych dát na základe malej vstupnej množiny
\item získaní výsledkov, ktoré sú si podobné, zároveň vyzerajú jedinečne
\item v jednoduchej tvorbe výstupov, ktoré požadujú pravideľnú alebo opakujúcu sa štruktúru
\end{itemize}

Svoje uplatnenie nachádza napr. vo filmoch{REF}. Moderné filmy sú často zasadené v fiktívnom svete, ktorý je dielom grafických umelcov. Generovanie uľahčuje prácu umelcov pri vytváraní modelov, ktoré sú si podobné, ako sú napr. modely vegetácie. {REF} Podobne je generovanie využitie pri rôznych filmárskych efektoch ako sú explózie, časticové efekty v podobe ohňa alebo hmly, ktoré by bolo zložité animovať manuálne.

Táto technika sa taktiež uplatňuje vo video hrách, kde sa využíva k tvorbe herných levelov, grafických modelov a ich textúr, prípade k tvorbe samotného príbehu.

V dnešnej dobe sa v oblasti generovania ustálili techniky, ktoré si následne rozoberieme.

\section{Šumy}

Obecne je šum N-rozmerný signál, ktorého hodnoty podliehajú istej náhodnosti. Typickým príkladom šumu sú náhodné a pseudonáhodné generátory čísiel, ktoré sa ale v počítačovej grafike nepoužívajú z dôvodu nespojitosti ich hodnôt. 

Spojité šumy majú v poč. grafike uplatnenie hlavne v oblasti vytváranie textúr.

Bežne sa používa napr. Perlinov šum, ktorého dvojrozmerná vizualizácia pripomína oblaky či dym. 

Okrem textúr je možné využiť šumy k tvorbe 3D terénu. V tomto prípade sa hodnoty šumu interpretujú ako výšková mapa.

TODO priklad výškovej mapy

\section{L-systémy}
L-systémy boli pôvodne predstavené v roku 1968 matematikom Lindenmayerom ako matematický formalizmus pre popis rastu a počítačovú vizualizáciu rastlín \cite{Lsystems}.
Ich definícia vychádza z prepisovacích systémov, ktoré upravujú o pararelné prepisovanie symbolov.
Napriek ich jednoduchosti umožňujú vytvárať komplexné štruktúry. V počítačovej grafike sa obvykle využívajú spolu s tzv. korytnačiou grafikou (angl. turtle graphics).

Existuje niekoľko rozšírení L-systémov, ktoré umožňujú elegantnejšie popísať zložité štruktúry. Jedným z nich sú parametrické L-systémy, ktoré si predstavíme v podkapitole
\ref{parametric}.


\subsection{Fraktály}
Fraktály sú objekty, ktoré sú si sami podobné. 
	- existujuce metody a principy, z ktorych vychadzaneme
	- co to vlastne je generovanie ? Co je to proceduralne generovanie ?
		Da sa formalne popisat ? Aké sú existujúce spôsoby ?
	- ako sa obvykle robi proc. generovanie (dostupne formalizmy)
	
