
%/////////////////////////////////////////////////
%OSNOVA BP PRACE (XDOBIA11)
%/////////////////////////////////////////////////

%01 Uvod
%====
\chapter{Úvod}
%	- k comu je mozne pouzit proc. gen (motivacia)
Táto práca sa zaoberá procedurálnym generovaním v oblasti počítačovej grafiky.
Procedurálne generovanie predstavuje jeden zo spôsobov akým je možné vytvárať vizuálne bohatý obsah na základe jednoduchých popisov. 
%	- kde je mozne taketo nieco vyuzit (hry, dema, kultura)
Podobné techniky v grafike majú svoje opodstatnenie. V počiatkoch prvých video hier našli svoje uplatnenie v generovaní obsahu, textúr a iný herných prvkov z dôvodu pamäťových obmedzení. 
V súčasnej dobe sa ich využitie spája s generovaním rozsiahlych trojrozmerných herných svetov.
%	- preco k tomu porebujeme prave jazyk (komfort, elegancia)

Pre využitie procedurálneho generovania je nutné zostrojiť systém, ktorý by autonómne vykonával generovanie z popisov. K tomuto účelu je možné
použiť obvyklé programovacie jazyky. Programovanie v týchto jazykoch ale ovykle obnáša nutnosť riešiť problémy ako sú správa pamäte a podobne, čo znižuje komfort programátora.
Preto má zmysel uvažovať o vytvorení nového programovacieho jazyka, ktorý umožní komfortne programovať.

%02 Teoria - Proc generovanie
%=========================
\chapter{L systémy}
%	L-SYSTEMY
%	=========
L-systémy boli pôvodne predstavené v roku 1968 matematikom Lindenmayerom ako matematický formalizmus pre popis rastu a počítačovú vizualizáciu rastlín \cite{Lsystems}.
Ich definícia vychádza z prepisovacích systémov, ktoré si predstavíme v podkapitole \ref{rewriting}.
Napriek ich jednoduchosti umožňujú vytvárať komplexné štruktúry. V počítačovej grafike sa obvykle využívajú spolu s tzv. korytnačiou grafikou (angl. turtle graphics).

Existuje niekoľko rozšírení L-systémov, ktoré umožňujú elegantnejšie popísať zložité štruktúry. Jedným z nich sú parametrické L-systémy, ktoré si predstavíme v podkapitole
\ref{parametric}.

\section{Prepisovacie systémy}\label{rewriting}
\section{DOL L-systémy}\label{dollsystems}
\section{Parametrické L-systémy}\label{parametric}
\section{Šumy}
%	SUMY (NOISE)
%	============
		- popis zakladnych sumov, ktore sa vyuzivaju v proc.gen

%03 Formalne jazyky a gramatiky
%==================
\chapter{Formálne jazyky a gramatiky}
	- uvod do matematickych formalizmov ktore sluzia na popis jazykov
%	REGULARNE JAZYKY
%	================
\section{Regulárne jazyky}
		- o co sa jedna, ako sa daju spracovat
%	BEZKONTEXTOVE JAZYKY
%	====================
\section{Bezkontextové jazyky}
		- parsery LL / LR
%	STRUKTURA PREKLADACOV
%	====================
\section{Štruktúra prekladačov}
		- lex->syntax->semantika

%03 Navrh jazyka
%============
\chapter{Návrh jazyka}
	- co vlastne ocakavame od naseho jazyka
	- cielova skupina
	- nastrelenie jazyka (co vlastne popisuje, klucove konstrukcie)
%	- LEXIKALNA ANALYZA
\section{Lexikálna analýza}
		- popis ako spracovavame vstupny text
		- zmienka o flexe	
%	- SYNTAKTICKA ANALYZA
\section{Syntaktická analýza}
%	- INTEPRET
\section{Interpret}
		- popis OO modelu
		- principy
	
