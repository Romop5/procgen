
%/////////////////////////////////////////////////
%OSNOVA BP PRACE (XDOBIA11)
%/////////////////////////////////////////////////

%01 Uvod
%====
\chapter{Úvod}
	- k comu je mozne pouzit proc. gen (motivacia)
	- kde je mozne taketo nieco vyuzit (hry, dema, kultura)
	- preco k tomu porebujeme prave jazyk (komfort, elegancia)

%02 Teoria - Proc generovanie
%=========================
\chapter{Procedurálne generovanie}
	- ako sa obvykle robi proc. generovanie (dostupne formalizmy)
	L-SYSTEMY
	=========
		- zakladny popis, niekolko ukazok
	SUMY (NOISE)
	============
		- popis zakladnych sumov, ktore sa vyuzivaju v proc.gen

%03 Formalne jazyky a gramatiky
%==================
\chapter{Formálne jazyky a gramatiky}
	- uvod do matematickych formalizmov ktore sluzia na popis jazykov
	REGULARNE JAZYKY
	================
		- o co sa jedna, ako sa daju spracovat
	BEZKONTEXTOVE JAZYKY
	====================
		- parsery LL / LR
	STRUKTURA PREKLADACOV
	====================
		- lex->syntax->semantika

%03 Navrh jazyka
%============
\chapter{Návrh jazyka}
	- co vlastne ocakavame od naseho jazyka
	- cielova skupina
	- nastrelenie jazyka (co vlastne popisuje, klucove konstrukcie)
	- LEXIKALNA ANALYZA
		- popis ako spracovavame vstupny text
		- zmienka o flexe	
	- SYNTAKTICKA ANALYZA
	- INTEPRET
		- popis OO modelu
		- principy
	
