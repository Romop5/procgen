%Kapitola by mala popisovat ako je jazyk implementovany:
%	- objektovy navrh = ake mame objekty
%	- gramaticke pravidla

\chapter{Implementácia}
Aby bolo možné programovací jazyk prakticky využiť, je nutné naprogramovať systém, ktorý spracuje textový vstup so zapísaným programom v sytaxi
jazyka a jeho činnosťou bude interpretácia vstupu.

%TODO Meduna ?
Čásť systému, ktorá realizuje spracovanie vstupu uživateľa sa obvykle nazýva \textit{lexikálny analyzátor} (\textit{ang. lexer}). Výstupom analyzátora je
séria štruktúr zvaných \textit{token}, ktoré predstavujú slovo s typom a hodnotou. 

V tomto projekte sú 

\section{Lexikálna analýza}
		- popis ako spracovavame vstupny text
		- zmienka o flexe	
Pre spracovanie vstupného textu je využitý voľne dostupný generátor Flex, šírený pod licenciou BSD. Flex umožňuje definovať tokeny
v komfortnej podobe a následne vygenerovať deterministický konečný automat, zapísaný v jazyku C alebo C++.

Vo Flexe sa jednotlivé tokeny definujú pomocou regulárneho výrazu. Prípadne kolízie regulárnych výrazov sú vrámci tohoto generátora
riešené chronológiou zoznamu pravidiel, teda skôr zadaná pravidlo má prioritu voči neskôr zadanému.

Pravidlá sa zadávajú v následujúcej forme:
\begin{verbatim}
"<REGULARNY-VYRAZ>" {<KOD>}
\end{verbatim}

\texttt{<REGULARNY-VYRAZ>} predstavuje reťazec, ktorého prečítaním zo vstupu sa vykoná kód \texttt{<KOD>}.
Výsledkom kódu je obvykle návratová hodnota, ktorá je predaná lexéru pomocou konštrukcie \texttt{return}.

Typ návratovej hodnoty je explicitne špecifikovať priamo v zdrojovom súbore s pravidlami, alebo importovať z výstupu Bisona.

%	- SYNTAKTICKA ANALYZA
\section{Syntaktická analýza}
Táto čásť projektu je založená na generátore syntaktického analyzátora GNU/Bison, ktorý pre zadanú gramatiku v BCN forme vytvorí preložiteľný
C/C++ kód, predstavujúci LALR sytaktický analyzátor.

Bison rozližuje medzi terminálnym a non-terminálnym symbolom. Terminálne symboly sú definované pomocou špecialneho príkazu a predstavujú
jednotlivé tokeny, získané lexérom.

Non-terminálne symboli sú definované pravidlami gramatiky. 
\subsection{Definícia pravidiel}
Pravidlá sú definované v následujúcom formáte:

\begin{verbatim}
<nazov-symbolu>:
        <symbol1> <symbol2> ... { } 
        | <symbol3> <symbol4> ... { }
\end{verbatim}

Pravidlá môžu mať viacero alternatív, ktoré sú v definícii oddelené pomocou znaku \texttt{|}. 



%	- INTEPRET

\section{Syntax}

Štruktúru zdrojového kódu je možné rozdeliť na definíciu typov, pravidiel a pomocných funkcií.

\subsection{Definícia typu}

\begin{verbatim}
<TYPE> <TYPENAME> { 
        <TYPE1> <NAME1>,	
        ...
        <TYPEN> <NAMEN>	
}
\end{verbatim}

\subsection{Definícia pravidla}

\begin{verbatim}
rule <TYPE> if { 
	return BOOLEAN;
} do {
}
\end{verbatim}

\section{Vstavané funkcie}

\begin{itemize}
\item \texttt{random()} \\
	Funkcia, ktorá slúži k výpočtu rovnomerného rozloženia, normalizovaného na interval $<0,1>$
\end{itemize}
\section{Modulárnosť}
\section{Prístup k stromu symbolov}

Navrhovaný jazyk obsahuje mechanizmus pre prácu s kontextom. Každý symbol obsahuje odkazy na symbol, z ktorého vznikol, ako aj symboli, ktoré sa vyskytli
pri prepisovaní v jeho okolí (kontext). Pre prístup k týmto symbolom sú dostupné vstavané nasledujúce vstavané funkcie:

\begin{itemize}
\item \texttt{getCurrentSymbolID()} \\
	Funkcia získa identifikátor práve derivovaného symbolu.
\item \texttt{getSiblingForID(ID,relativePos)} \\
	Funkcia vráti identifikátor symbolu, ktorý sa nachádza v rovnakom derivačnom reťazci ako symbol \texttt{ID}. \texttt{relativePos} určuje vzdialenosť
	od referečného symbolu. Kladné hodnoty predstavujú vzdialenosť z prava, záporné z ľava. 
	V prípade, že symbol neexistuje vráti funkcia \texttt{UNKID}
\item \texttt{getParent(ID)} \\
	Získa identifikátor symbolu, z ktorého vznikol symbol \texttt{ID}.
\end{itemize}

Nakoľko je náš navrhovaný jazyk typovaný a symbol môže obecne vzniknúť z ľubovoľného symbolu obecne odlišného typu, je nutné zabezpečiť manipuláciu so
symbolom, ktorého typ dopredu nepoznáme.

K tomuto slúžia nasledujúce funkcie:
\begin{itemize}
\item \texttt{hasPropertyWithName(ID,property)} \\
	Funkcia vracia hodnotu \texttt{TRUE} v prípade ak symbol, resp. jeho typ obsahuje vlastnosť \texttt{property}. Inak \texttt{FALSE}.
\item type \texttt{getPropertyAsType(ID,property,type)}

\end{itemize}


