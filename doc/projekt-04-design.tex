%03 Navrh jazyka
%============
\chapter{Návrh jazyka}

Formálna báza navrhovaného jazyka vychádza z fungovania \textit{L-systémov}.
Základom jazyka sú symboly a pravidlá, ktorými sa odvodzujú nové symboly.


Fungovanie programu popísaného našim jazykom je založené na prepisovaní typovaných symbolov.
Po spustení programu sa vytvorí uživateľom definovaný počiatočný reťazec symbolov s počiatočnými hodnotami.
Následne prebiehajú takzvané prepisovacie cyklusy. V každej iterácii sa postupne pre každý symbol aktuálneho reťazca
vyhodnotia všetky definované pravidlá pre jeho typ a z množiny prípustných pravidieľ sa náhodne vyberie.
Vybraté pravidlo je aplikované a jeho aplikáciou môže vniknúť nový symbol, ktorý sa pridá na koniec reťazca pre ďalšiu iteráciu.
V prípade, že žiadne z pravidieľ nie je aplikovateľné, je tento symbol taktiež pridaný k ďalšej iterácii.

Výpočet je ukončený v momente keď počas iterácie nebolo použité ani jedno z uživateľových pravidiel.

\section{Symboly}
Symboly sú definované uživateľom, pričom sa jedná o štruktúry, ktoré sú odvodené z predom definovaných
typov a vstavaných (atomických typov). Jednotlivé zložky (vlastnosti) je možné unikátne pomenovať.

\section{Pravidlá a funkcie}
Uživateľ taktiež špecifikuje pravidlá. Každé pravidlo je zložené z dvoch časti: \textbf{predikátovej funkcie} a \textbf{procedúry}.

Predikátová funkcia zobrazuje parametre symbolu na boolovskú hodnotu, ktorá určuje, či je pravidlo aplikovateľné na
aktuálny symbol. 

Precedúra pravidla predstavuje akciu, resp. súbor akcií, ktoré sa majú vykonať pri aplikácii pravidla na symbol. 

Pre zvýšenie komfortu je rovnako možné definovať vlastné pomocné funkcie, ktoré je následne možné následne použiť v iných funkciach. 
Jazyk ale striktne nepovoluje rekurziu pri volaní funkcií.

Vyžšie spomenuté konštrukcie definujú svoju činnosť na základe sekvencie príkazov. Príkazy v navrhovanom jazyku predstavujú \textit{imperatívnu} časť jazyka.
Každý príkaz je buď priradenie, podmienka, cyklus alebo výraz.

\section{Syntax}

Štruktúru zdrojového kódu je možné rozdeliť na definíciu typov, pravidiel a pomocných funkcií.

\subsection{Definícia typu}

\begin{verbatim}
<TYPE> <TYPENAME> { 
        <TYPE1> <NAME1>,	
        ...
        <TYPEN> <NAMEN>	
}
\end{verbatim}

\subsection{Definícia pravidla}

\begin{verbatim}
rule <TYPE> if { 
	return BOOLEAN;
} do {
}
\end{verbatim}

\section{Vstavané funkcie}

\begin{itemize}
\item \texttt{random()} \\
	Funkcia, ktorá slúži k výpočtu rovnomerného rozloženia, normalizovaného na interval $<0,1>$
\end{itemize}
\section{Modulárnosť}
\section{Prístup k stromu symbolov}

Navrhovaný jazyk obsahuje mechanizmus pre prácu s kontextom. Každý symbol obsahuje odkazy na symbol, z ktorého vznikol, ako aj symboli, ktoré sa vyskytli
pri prepisovaní v jeho okolí (kontext). Pre prístup k týmto symbolom sú dostupné vstavané nasledujúce vstavané funkcie:

\begin{itemize}
\item \texttt{getCurrentSymbolID()} \\
	Funkcia získa identifikátor práve derivovaného symbolu.
\item \texttt{getSiblingForID(ID,relativePos)} \\
	Funkcia vráti identifikátor symbolu, ktorý sa nachádza v rovnakom derivačnom reťazci ako symbol \texttt{ID}. \texttt{relativePos} určuje vzdialenosť
	od referečného symbolu. Kladné hodnoty predstavujú vzdialenosť z prava, záporné z ľava. 
	V prípade, že symbol neexistuje vráti funkcia \texttt{UNKID}
\item \texttt{getParent(ID)} \\
	Získa identifikátor symbolu, z ktorého vznikol symbol \texttt{ID}.
\end{itemize}

Nakoľko je náš navrhovaný jazyk typovaný a symbol môže obecne vzniknúť z ľubovoľného symbolu obecne odlišného typu, je nutné zabezpečiť manipuláciu so
symbolom, ktorého typ dopredu nepoznáme.

K tomuto slúžia nasledujúce funkcie:
\begin{itemize}
\item \texttt{hasPropertyWithName(ID,property)} \\
	Funkcia vracia hodnotu \texttt{TRUE} v prípade ak symbol, resp. jeho typ obsahuje vlastnosť \texttt{property}. Inak \texttt{FALSE}.
\item type \texttt{getPropertyAsType(ID,property,type)}

\end{itemize}


\section{Objektový návrh}


