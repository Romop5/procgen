%03 Navrh jazyka
%============
\chapter{Návrh jazyka}

Jazyk, ktorý bol navrhnutý v tejto práci vychádza vo svojom základe z \textbf{L-systémov}.
Základom jazyka sú symboly a pravidlá, ktorými sa odvodzujú nové symboly.


\section{Definícia jazyka}
Fungovanie programu popísaného našim jazykom je založené na prepisovaní typovaných symbolov.
Po spustení programu sa vytvorí uživateľom definovaný počiatočný reťazec symbolov s počiatočnými hodnotami.
Následne prebiehajú takzvané prepisovacie cyklusy. V každej iterácii sa postupne pre každý symbol aktuálneho reťazca
vyhodnotia všetky definované pravidlá pre jeho typ a z množiny prípustných pravidieľ sa náhodne vyberie.
Vybraté pravidlo je aplikované a jeho aplikáciou môže vniknúť nový symbol, ktorý sa pridá na koniec reťazca pre ďalšiu iteráciu.
V prípade, že žiadne z pravidieľ nie je aplikovateľné, je tento symbol taktiež pridaný k ďalšej iterácii.

Výpočet je ukončený v momente keď počas iterácie nebolo použité ani jedno z uživateľových pravidiel.

\subsection{Symboly}
Symboly sú typované. Každý uživateľom definovaný symbol predstavuje unikátny typ, ktorý predstavuje štruktúru, zloženú z atomických (preddefinovaných typov) alebo
z iných kompozitných typov. Jednotlivé zložky majú vrámci štruktúry unikátny názov, ktorý predstavuje ich identifikáciu.

\subsection{Pravidlá a funkcie}
Uživateľ taktiež špecifikuje pravidlá. Každé pravidlo je zložené z dvoch časti: \textbf{predikátovej funkcie} a \textbf{procedúry}.

Predikátová funkcia zobrazuje parametre symbolu na boolovskú hodnotu, ktorá určuje, či je pravidlo aplikovateľné na
aktuálny symbol. 

Precedúra pravidla predstavuje akciu, resp. súbor akcií, ktoré sa majú vykonať pri aplikácii pravidla na symbol. 

Pre zvýšenie komfortu je rovnako možné definovať vlastné pomocné funkcie, ktoré je následne možné následne použiť v iných funkciach. 
Jazyk ale striktne nepovoluje rekurziu pri volaní funkcií.

Vyžšie spomenuté konštrukcie definujú svoju činnosť na základe sekvencie príkazov. Príkazy v navrhovanom jazyku predstavujú \textit{imperatívnu} časť jazyka.
Každý príkaz je buď priradenie, podmienka, cyklus alebo výraz.


\section{Objektový návrh}
Pred samotnou implementáciou jazyka je nutné premyslieť ako budú jednotlivé štruktúry uložené v pamäti a ako budú vyzerať ich vzájomné väzby.
\subsection{Symboly}
Ako už bolo zmienené, symboly sú kompozitné typy. Je teda nutné vytvoriť rozhranie, ktoré umožňí definovať atomické aj zložené typy a pre tento účel je vhodné využiť návrhový vzor \textit{Composite} (TODO).  Rovnako uchovanie samotného typu vyžaduje uložiť metadáta typu (jednoznačný identifikátor, názov, prípadne podtypy) a samotné inštancie typov. 

Hodnoty typov sú reprezentované abstraktnou triedou \texttt{Resource}, ktorá uchováva identifikáciu typu a typ typu. Pomocou dedičnosti sa následne rozlišuje medzi atomickým (\texttt{AtomicResource} a kompozitným typom \texttt{CompositeResource}.

Z hľadiska uloženia metadát sa javí neefektívne ukladať popis typu v každej inštancii typu. Preto je nutné zaviesť akýsi register, ktorý by uchoval popis typu na jednom mieste. Z toho dôvodu je pridaná trieda \textit{TypeRegister}, ktorej účelom je spravovať definované typy ako aj ponúkať mechizmus k tvorbe inštancii takýchto typov. 
Pre každý typ obsahuje tento register inštanciu triedy \textit{TypeDecriptor}. 

\begin{figure}[ht]
\caption{Lol}
\centering
\begin{tikzpicture}[show background grid]
	\begin{abstractclass}[text  width=6cm]{Resource}{0,0}
		\attribute{- typeId}
		\operation{getTypeId()}
	\end{abstractclass}
	\begin{class}[text  width=4cm]{AtomicResource}{-4,3}
		\inherit{Resource}
		\attribute{- data}
		\operation{getTypeId()}
		\operation{getData(): void*}
	\end{class}
	\begin{class}[text  width=4cm]{CompositeResouce}{4,3}
		\inherit{Resource}
		\operation{getTypeId()}
		\operation{getComponent(id): AbstractResource}
	\end{class}

	\aggregation{CompositeResouce}{components}{n}{AtomicResource}
\end{tikzpicture}
\end{figure}

\begin{tikzpicture}
	\begin{class}[text  width=6cm]{Statement}{0,0}
		\operation{operator()() : bool returned}
	\end{class}
	\begin{class}[text  width=4cm]{If}{0,3}
		\inherit{Statement}
		\operation{operator()() : bool returned}
	\end{class}
	\begin{class}[text  width=4cm]{Body}{3,3}
		\operation{operator()() : bool returned}
	\end{class}
\end{tikzpicture}
